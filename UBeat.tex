\documentclass[11pt]{article}

\usepackage[utf8]{inputenc}
\usepackage[T1]{fontenc}
\usepackage[french]{babel}
\usepackage{enumitem}

\title{UBeat: Web development project}
\author{Team 11}
\date{1\up{st} October 2017}

\begin{document}

    \maketitle
    \newpage
    
    \section{Introduction}
    
        \subsection{Project}
        
        \noindent
        The UBeat project is a web application created for the GLO-3102 class. Its main goal is to create
        a music platform where a user can create playlists and share them with friends.
    
        \subsection{Team}
        
        \noindent
        Our team is composed of 5 people: \\
        
<<<<<<< HEAD
        \subsection{Links}
		\noindent
		Page Album: the name of the Artist (Black eyed peas) go to Artist's page.
     	\\
        \\
        Page Artist: the name of the album (The Beginning) go to Album page.
        \\
        \\

=======
>>>>>>> e645be82119c42e19a88a5f158df713ab981fc47
        \begin{description}[leftmargin=*]
        \item[Enora Le-Cavorzin] (enlec, 111191036)
        \item[Rémi Weislinger] (rewei2, 111191060)
        \item[Louis-Philippe Dubuc] (lpdub7, 111118879)
        \item[Guillaume Binet] (gubin1, 903195070)
        \item[Lionel Karmes] (likar7, 111190966)
        \end{description}
        
        

    \section{Getting started}
    
        \subsection{Setup}
    
        \noindent
        Well, you just have to \texttt{ npm install } , and \texttt{ npm start } afterwards.
        
        \subsection{Links}
        
		\noindent
		On the home page, you will find the links to the artist page and the album page at the bottom. \\
		On the artist page, you can click on the album "The Beginning" to reach the album page. \\
		On the album page, you can click on the artist name under the album name to reach the artist page.

\end{document}