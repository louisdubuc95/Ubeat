\documentclass[11pt]{article}

\usepackage[utf8]{inputenc}
\usepackage[T1]{fontenc}
\usepackage[french]{babel}
\usepackage{enumitem}

\title{UBeat: Web development project}
\author{Team 11}
\date{December 17\up{th} 2017}

\begin{document}

    \maketitle
    \newpage

    \section{Introduction}

        \subsection{Project}

        \noindent
        The UBeat project is a web application created for the GLO-3102 class. Its main goal is to create
        a music platform where a user can create playlists and share them with friends.

        \subsection{Team}

        \noindent
        Our team is composed of 5 people: \\

        \begin{description}[leftmargin=*]
        \item[Enora Le-Cavorzin] (enlec, 111191036)
        \item[Rémi Weislinger] (rewei2, 111191060)
        \item[Louis-Philippe Dubuc] (lpdub7, 111118879)
        \item[Guillaume Binet] (gubin1, 903195070)
        \item[Lionel Karmes] (likar7, 111190966)
        \end{description}

    \section{Getting started}

        \subsection{Setup}

        \noindent
        Well, you just have to \texttt{ npm install } , and \texttt{ npm start } afterwards.

        \subsection{Content}
		\noindent
		Navigation bar:
		\begin{itemize}
		\item  Click on "Home" to go on Home page ;
		\item  Click on "Browse playlists" to go on playlists' page - for creating and searching one ;
		\item The input is for the global search ;
		\item When signed in : "My profile" button go to the profile of signed id person ;
		\item Sign in can be use to display sign up or sign in modal.
		\end{itemize}		
		\noindent \\
		Sign in modal:
		\begin{itemize}
		\item By default it displays sign in modal ;
		\item One can reach sign up by clicking "Create an account" ;
		\item Sign in : write a valid email and password then click "login"  ;
		\item Sign up : write a username, a valid email and a password then click "Create".
		\end{itemize}
		\noindent \\
		On the home page:
		\begin{itemize}
		\item Click on "View our artist page" to reach an example artist page (id in the url) ;
		\item Click on "View our album page" to reach an example album page (id in the url).
		\end{itemize}
		\noindent \\
		On the artist page:
		\begin{itemize}
		\item Click on the "Listen on apple music" button to reach the artist page on Apple Music ;
        \item Click on any album to reach a new page and see the tracklist, listen to samples, ...
        \end{itemize}

        \noindent \\
        On the album page:
        \begin{itemize}
        \item Click on the artist name to go back to the artist page.
        \item Click on the "Listen on apple music" button to reach the artist page on Apple Music.
        \item Click on "Add to playlist" and choose a playlist to add the entire album to it.
            \begin{itemize}
            \item Click again on the button to cancel
            \end{itemize}
        \item Click on a "Play" button to listen to a short music sample.
        \item Click on a "Plus" button and choose a playlist to add a song to it.
            \begin{itemize}
            \item Click again on the button to cancel
            \end{itemize}
        \end{itemize}
        
        \noindent \\
        On the playlists page:
        \begin{itemize}
     	\item Input on top right is for searhing ;
        \item Input with placeholder "Choose a name" : one can write the name of the playlist to create ;
   		\item When signed in, one can see playlists of users ;
        \item Choose a name and click on "Create a new playlist" to create one. Its name cannot be empty.
        \item Click on a playlist's name to display its tracks. Nothing happens if there aren't any.
            \begin{itemize}
            \item Click on a track's "Trash" button to remove it from the playlist
            \item Other track controls are identical to the ones on the album page.
            \end{itemize}
        \item Click on the "Pencil" button to rename a playlist. Its name cannot be empty.
            \begin{itemize}
            \item Press enter to apply
            \item Press escape or click again on the button to cancel
            \end{itemize}
        \item Click on the "Trash" button to delete a playlist
        \end{itemize}
	    \noindent \\
		On profile page:
	    \begin{itemize}
    	\item Input on top right is for searhing a user ;
    	\item The figure with a "+" is to add this user to following ;
    	\item The figure with a "x" is to stop following this user ;
    	\item Refer to playlists part for the playlist on the bottom.
        \end{itemize}
    	

	\section{Special features}
	\noindent
	When a biography and/or a picture of the group is on LastFM databases, they will be displayed on artist's page.
	\\
	\\
	In profile and users' search, we can see gravatar. A default picture if one doesn't have a gravatar image or the real picture.

\end{document}
